% Options for packages loaded elsewhere
\PassOptionsToPackage{unicode}{hyperref}
\PassOptionsToPackage{hyphens}{url}
%
\documentclass[
]{article}
\usepackage{amsmath,amssymb}
\usepackage{lmodern}
\usepackage{ifxetex,ifluatex}
\ifnum 0\ifxetex 1\fi\ifluatex 1\fi=0 % if pdftex
  \usepackage[T1]{fontenc}
  \usepackage[utf8]{inputenc}
  \usepackage{textcomp} % provide euro and other symbols
\else % if luatex or xetex
  \usepackage{unicode-math}
  \defaultfontfeatures{Scale=MatchLowercase}
  \defaultfontfeatures[\rmfamily]{Ligatures=TeX,Scale=1}
\fi
% Use upquote if available, for straight quotes in verbatim environments
\IfFileExists{upquote.sty}{\usepackage{upquote}}{}
\IfFileExists{microtype.sty}{% use microtype if available
  \usepackage[]{microtype}
  \UseMicrotypeSet[protrusion]{basicmath} % disable protrusion for tt fonts
}{}
\makeatletter
\@ifundefined{KOMAClassName}{% if non-KOMA class
  \IfFileExists{parskip.sty}{%
    \usepackage{parskip}
  }{% else
    \setlength{\parindent}{0pt}
    \setlength{\parskip}{6pt plus 2pt minus 1pt}}
}{% if KOMA class
  \KOMAoptions{parskip=half}}
\makeatother
\usepackage{xcolor}
\IfFileExists{xurl.sty}{\usepackage{xurl}}{} % add URL line breaks if available
\IfFileExists{bookmark.sty}{\usepackage{bookmark}}{\usepackage{hyperref}}
\hypersetup{
  pdftitle={Amortiguadores de masa sintonizados - Parte 1},
  pdfauthor={Luis Damián García Vega, José de Jesús García Piña, Gabriel Peytral Borja},
  hidelinks,
  pdfcreator={LaTeX via pandoc}}
\urlstyle{same} % disable monospaced font for URLs
\usepackage[margin=1in]{geometry}
\usepackage{graphicx}
\makeatletter
\def\maxwidth{\ifdim\Gin@nat@width>\linewidth\linewidth\else\Gin@nat@width\fi}
\def\maxheight{\ifdim\Gin@nat@height>\textheight\textheight\else\Gin@nat@height\fi}
\makeatother
% Scale images if necessary, so that they will not overflow the page
% margins by default, and it is still possible to overwrite the defaults
% using explicit options in \includegraphics[width, height, ...]{}
\setkeys{Gin}{width=\maxwidth,height=\maxheight,keepaspectratio}
% Set default figure placement to htbp
\makeatletter
\def\fps@figure{htbp}
\makeatother
\setlength{\emergencystretch}{3em} % prevent overfull lines
\providecommand{\tightlist}{%
  \setlength{\itemsep}{0pt}\setlength{\parskip}{0pt}}
\setcounter{secnumdepth}{-\maxdimen} % remove section numbering
\ifluatex
  \usepackage{selnolig}  % disable illegal ligatures
\fi

\title{Amortiguadores de masa sintonizados - Parte 1}
\author{Luis Damián García Vega, José de Jesús García Piña, Gabriel
Peytral Borja}
\date{09/12/2021}

\begin{document}
\maketitle

\newcommand{\bcenter}{\begin{center}}
\newcommand{\ecenter}{\end{center}}

\begin{center}

\hypertarget{proyecto-final-de-ecuaciones-diferenciales-ordinarias}{%
\section{Proyecto final de Ecuaciones Diferenciales
Ordinarias:}\label{proyecto-final-de-ecuaciones-diferenciales-ordinarias}}

\hypertarget{amoritiguadores-de-masa-sintonizados}{%
\section{Amoritiguadores de masa
sintonizados}\label{amoritiguadores-de-masa-sintonizados}}

\end{center}

\hypertarget{introducciuxf3n}{%
\subsection{\texorpdfstring{\textbf{Introducción}}{Introducción}}\label{introducciuxf3n}}

Los amortiguadores de masa sintonizada son masas vibrantes que se mueven
fuera de fase con el movimiento de alguna estructura en la que se está
suspendida, generalmente se instalan en la parte superior de los
edificios y que al estar en sintonía con la frecuencia de movimiento del
edificio funcionan como un contrapeso que oscila y contrarresta el
movimiento del edificio disipando la energía de un terremoto o del
viento.\\
Esta no es la única aplicación práctica que tienen, también pueden ser
usados en mesas de cirugía para evitar el movimiento de estas durante
una intervención quirurjica de alta precisión.\\
La eficacia de disipación de energía de un TMD depende de la precisión
de su sintonía, el tamaño de su masa en comparación con la masa modal de
su modo objetivo, es decir, su relación de masa, y la extensión de la
amortiguación interna incorporada en el amortiguador de masa
sintonizado.\\
Básicamente, uno puede pensar en un TMD como un contrapeso mecánico para
una estructura que consta de una masa en movimiento (aproximadamente 1 a
2\% de la masa de la estructura).

\textbf{Modelado de un amortiguador de masa sintonizado} Representamos
la rigidez de la masa original de un sistema usando un resorte y la
absorción de energía usando un amortiguador. Adjuntaremos un resorte y
un amortiguador (TMD) a la masa original y exploraremos qué tanto se
puede reducir las vibraciones debidas a fuerzas externas. Se
caracterizará el desplazamiento de la masa sobre una posición de
equilibrio inicial por un solo grado de libertad. Sin embargo, en el
caso del TMD, agregaremos una segunda masa y un resorte a la masa
primaria. El desplazamiento de la masa TMD se caracterizará por un grado
separado de libertad sobre su propia posición de equilibrio inicial. Por
lo tanto, un TMD es un par de osciladores armónicos amortiguados
acoplados.\\
Una gran masa \(m_1\) en un ``resorte'' con constante de resorte \(k_1\)
se acopla a una masa más pequeña \(m_2\) por un resorte con constante de
resorte \(k_2\) y un amortiguador con coeficiente de amortiguación
\(c_2\). La gran masa también podría estar naturalmente amortiguada con
un coeficiente de amortiguación o de restauración \(c_1\). Para nuestro
estudio, tomamos \(c_1 = 0\) y \(c_2 = 0\). Es decir, supondremos que no
hay resistencia ni amortiguamiento de la masa original o del TMD. Al
ajustar los valores de \(m_2\) y \(k_2\) la amplitud máxima de las
oscilaciones de \(m_1\) se puede reducir.\\
En este caso sin resistencia (\(c_1 = 0\) y \(c_2 = 0\)) cuando
aplicamos una carga externa o fuerza impulsora, \(f(t)\), tenemos el
siguiente modelo de ecuación diferencial de un solo grado de libertad
para el desplazamiento, \(y(t)\), de nuestra masa desde su equilibrio
estático, es decir, un sistema de resorte-masa (sin salpicadero):
\[my''(t) + ky(t) = f(t), y(0)=y_0, y'(0) = v_0; (1)\] La frecuencia
natural de masa \(m_1\) se puede obtener de la solución de la porción
homogénea,(2), de nuestra ecuación diferencial (1),

\[my''(t) + ky(t) = 0, \ y(0)=y_0, \  y'(0) = v_0; \ (2)\] De (3) vemos
que la frecuencia natural de la solución no impulsada u homogénea de (2)
es \$\omega\_0 = \sqrt \frac {k}{m} \$ mientras que la solución
homogénea completa para (1) es:
\[y(t)= \frac {v_0 \sqrt m \sin(\sqrt \frac {k}{m}t)+y_0 \sqrt k \cos(\sqrt \frac {k}{m} t)}{\sqrt k}, \ (3) \]

\hypertarget{actividad-1}{%
\subsubsection{\texorpdfstring{\emph{Actividad
1}}{Actividad 1}}\label{actividad-1}}

Usando la Segunda Ley de Newton, que dice que la masa multiplicada por
la aceleración de esa masa es igual a la suma de todas las fuerzas
externas que actúan sobre la masa, y la Figura 4 como fuente del
Diagrama de cuerpo libre,muestran que las ecuaciones que gobiernan los
movimientos de la estructura y el amortiguador están dadas por:

\[ m_1 \space x''_1(t) = -k_1 x_1(t)- c_1x'_1(t)-k_2(x_1(t)-x_2(t))-c_2(x'_1(t)-x'_2(t))+p_0 \space cos(\omega t), \space\space\space\space (1)\]
\[ m_2 \space x''_2(t) = -k_2(x_2(t)-x_1(t))-c_2(x'_2(t)-x'_1(t)), \space\space\space\space (2) \]

\hypertarget{actividad-2}{%
\subsubsection{\texorpdfstring{\emph{Actividad
2}}{Actividad 2}}\label{actividad-2}}

\begin{enumerate}
\def\labelenumi{\alph{enumi})}
\tightlist
\item
  Determine la forma de las ecuaciones de la actividad anterior cuando
  \(c_1 = 0\) y \(c_2 = 0\).
\item
  En el sistema resultante use \(m_1 = 10\) y \(k_1 = 90\) con
  \(y(0) = 0\) y \(y''(0) = 0\), mientras que
  \(f(t) =10 \space\space \cos (3t)\) y establezca \(m_2 = 1\) con
  \(k_2 = 0\), es decir, retirando el segundo sistema de masas. Explique
  los resultados de importancia física e impacto de los números 10, 90 y
  3 en la oracióna nterior en términos del movimiento del oscilador
  inicial.
\item
  Resuelva la ecuacion diferencia simple resultante para el movimiento
  de la masa \(m_1\) durante un intervalo de tiempo de 20 unidades y
  grafique el desplazamiento de esa masa, \(x_1(t)\), durante ese
  intervalo de tiempo. Explique las observaciones.
\item
  En las ecuaciones de la actividad anterior use los valores \(m_1=10\)
  y \(k_1=90\) con \(y(0)=0\) y \(y'(0)=0\) mientras
  \(f(t)=10 \space\space \cos (3t)\) y establezca \(m_2=1\) con valores
  variables de \(k_2\). Mantenga \(c_1=0\) y \(c_2=0\). ¿Qué observa en
  la amplitud máxima de la masa inicial \(m_1\) cuando se cambia
  \(k_2\)? Defiende tu observación con datos y/o gráficas.
\item
  De (d) ¿cuál es el ``mejor'' valor de \(k_2\)? Asegúrese de definir la
  palabra ``mejor''.
\item
  Para obtener el mejor valor de \(k_2\), determine el desplazamiento de
  amplitud máximo para la masa \(m_1 ,\space\space x_1 (t)\)en un rango
  de frecuencias
\end{enumerate}

\hypertarget{actividad-3}{%
\subsubsection{\texorpdfstring{\emph{Actividad
3}}{Actividad 3}}\label{actividad-3}}

Consideremos otra configuración comparable a la Actividad 2 para la
práctica. Observe que la masa \(m_2\) es solo el 1\% de la masa \(m_1\),
lo cual es bastante realista en el diseño estructural cuando se utilizan
TMD. a) Determine la forma de las ecuaciones de la actividad anterior
cuando \(c_1 = 0\) y \(c_2 = 0\) b) En el sistema resultante use
\(m_1 = 10\) y \(k_1 = 90\) con \(y(0) = 0\) y \(y'(0) = 0\) mientras
\(f(t) =10 \space\space \cos (3t)\) y ajuste \(m_2 = 0.1\) con
\(k_2 = 0\), es decir, retire el segundo sistema de masa. Explique la
importancia física e impacto de los números 10, 90 y 3 en la oración
anterior en términos del movimiento del oscilador inicial. c) Resuelva
la ecuación diferencial simple resultante para el movimiento de la masa
\(m_1\) durante un intervalo de tiempo de 20 unidades y grafique el
desplazamiento de esa masa, \(x_1 (t)\), durante ese intervalo de
tiempo. Explique las observaciones. d) En las ecuaciones de la actividad
anterior use los valores \(m_1=10\) y \(k_1=90\) con \(y(0)=0\) y
\(y'(0)=0\) mientras \(f(t)=5 \space\space \cos (3t)\) y establezca
\(m_2=0.1\) con valores variables de \(k_2\). Mantenga \(c_1=0\) y
\(c_2=0\). ¿Qué observa en la amplitud máxima de la masa inicial \(m_1\)
cuando se cambia \(k_2\)? Defiende tu observación con datos y/o
gráficas. e) De (d) ¿cuál es el ``mejor'' valor de \(k_2\)? Asegúrese de
definir la palabra ``mejor''. f) Para obtener el mejor valor de \(k_2\),
determine el desplazamiento de amplitud máximo para la masa
\(m_1 ,\space\space x_1 (t)\)en un rango de frecuencias

\hypertarget{actividad-4}{%
\subsubsection{\texorpdfstring{\emph{Actividad
4}}{Actividad 4}}\label{actividad-4}}

Del material introductorio para este escenario ofrecido anteriormente y
los análisis en las Actividades 2 y 3 realiza una descripción de cómo
construir un TMD para detener los fenómenos de resonancia en el caso de
las siguientes ecuaciones, que son el caso de las ecuaciones de la
actividad 1 donde no hay amortiguamiento, es decir, \(c_1 = 0\) y
\(c_2 = 0\):
\[ m_1 \space x''_1(t) = -k_1 x_1(t)-k_2(x_1(t)-x_2(t))+p_0 \space cos(\omega t), \space\space\space\space (1)\]
\[ m_2 \space x''_2(t) = -k_2(x_2(t)-x_1(t)), \space\space\space\space (2) \]

\end{document}
